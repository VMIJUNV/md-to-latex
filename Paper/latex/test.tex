\documentclass{ctexart}

\usepackage{amsmath}
\usepackage{amssymb}
\usepackage{enumitem}
\usepackage{booktabs}
\usepackage{tabularx}
\usepackage{graphicx}
\usepackage{geometry}
\usepackage{float}
\usepackage{listings}
\usepackage{fancyhdr}
\usepackage{csvsimple}
\usepackage{hyperref}

\geometry{
    a4paper,
    left=20mm,
    right=20mm,
    top=20mm,
    bottom=20mm
}

\pagestyle{fancy}
\renewcommand{\headrulewidth}{0pt}
\fancyhead{}
\fancyfoot[C]{\thepage}

\newcommand{\rootpath}{.}

\begin{document}



\section{介绍}

Word和LaTeX是论文写作常用的排版软件,二者各有优缺点,没有优劣之分。但是笔者作为一名理科生,更偏向于用LaTeX来排版,首先LaTeX对数学公式的支持更完善,排版的效果优美,其次是LaTeX分离了内容与形式,排版逻辑清晰,最后LaTeX是以文本形式保存的,文章的管理和编辑轻便。

但是LaTeX的语法复杂,编译卡顿,源代码的阅读性也不好,直接使用LaTeX来写文章并不是一个舒服的选择。与此同时Markdown作为一种标记语言,语法简单、易于学习、轻量简洁,更容易让作者专注于内容的创作,所以笔者认为前期使用Markdown创作内容,后期使用LaTeX调整格式是一个不错的选择。

目前可以使用软件pandoc将Markdown转化为LaTeX,笔者我认为这个方法实现的效果不够优雅,功能也比较少。

为此笔者亲自设计了一个转换方案,这套方案有以下优点和特点:

\begin{enumerate}
  \item {\textbf{模板化:} 通过预定义的模板,用户可以提前设定文章的格式,使得Markdown文本可以专注于内容创作,而无需担心格式问题。}\item {\textbf{图表引用:} 在Markdwon中可以利用链接快捷方便的调用图片和表格文件,本方案会自动处理这些文件,并将其转化为LaTeX格式。}\item {\textbf{嵌入LaTeX源码:} 支持在Markdown中嵌入LaTeX代码,这让本方案有了极大的灵活性,实现论文引用、交叉引用等复杂的排版功能。}\item {\textbf{定制化:} 本项目的设计考虑到了用户可能的不同需求,部分功能可以根据用户的实际需求进行简单的代码修改,以实现更多定制化功能。}\item {\textbf{简洁优雅:} 本项目生成的LaTeX代码简洁优雅,易于阅读和维护。}
\end{enumerate}
\section{演示}

\subsection{段落标题列表}

演示文本\textbf{演示文本}演示文本\emph{example}演示文本

\begin{itemize}
  \item {无序列表}\item {无序列表}
\end{itemize}
\begin{enumerate}
  \item {有序列表1}\item {有序列表2}
\end{enumerate}
\subsection{公式}

行内公式: $\mathrm{e}^{x}=1+x+\frac{x^{2}}{2 !}+\cdots+\frac{x^{n}}{n !}+o\left(x^{n}\right)$。

行间公式:

\begin{equation}
  f(x)=f\left(x_{0}\right)+f^{\prime}\left(x_{0}\right)\left(x-x_{0}\right)+\frac{f^{\prime \prime}\left(x_{0}\right)}{2 !}\left(x-x_{0}\right)^{2}+\cdots
\end{equation}
\subsection{图表}

\begin{figure}[H]
    \centering
    \includegraphics[width=0.4\textwidth]{\rootpath/Data/测试图1.jpeg}
    \caption{图片测试1}
    \label{图片测试1}
\end{figure}

\href{/Data/测试图2.jpg}{图}
\begin{table}[H]
    \centering
    \begin{tabular}{ccccc}
        \toprule
        第一列 & 第二列 & 第三列 & 第四列 & 第五列\\
        \\midrule
        1 & 2 & 3 & 4 & 5\\
        \bottomrule
    \end{tabular}
\end{table}
\subsection{嵌入Latex代码}

引用图\ref{图片测试1}。

引用论文\cite{lu_deepxde_2021}。

自动生成参考文献。
\newpage
\bibliographystyle{plain}
\bibliography{\rootpath/Data/docu.bib}



\end{document}